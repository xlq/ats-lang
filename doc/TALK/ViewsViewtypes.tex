%%%%%%
%%
%% Title: Views and Viewtypes in ATS
%% Author: Hongwei Xi, Boston University
%%
%% Date: December 5th, 2011
%% Time: 4-5PM
%%
%%%%%%
%%
%% Abstract
%%
%% ATS is a statically typed general-purpose programming language.  The
%% signatory feature of ATS is a programming paradigm named
%% programming-with-theorem-proving in which code for (run-time) computation
%% and code for proof construction can be written in a syntactically
%% intertwined manner.  In ATS, there is direct support for both dependent
%% types and linear types. While the dependent types in ATS, which are of a
%% restricted form originated from the development of Dependent ML (DML), are
%% well-studied, the linear types in ATS are much less well-known.  In this
%% talk, I will give an introduction to a notion of linear types referred to
%% as viewtypes in ATS, which combine views (i.e., linear types for proofs)
%% and types (for run-time values). In addition, I will present several
%% concrete examples to illustrate certain practical uses of views and
%% viewtypes.
%%
%%%%%%
%%
%% Slide 1
%%
%%%%%%
%%
%% Slide 2
%%
%%%%%%

%%%%%% end of [ViewsViewtypes.tex] %%%%%%

